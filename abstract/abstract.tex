\documentclass[12pt, letterpaper]{article}
\usepackage[utf8]{inputenc}
\usepackage{caption} % for table captions
\usepackage{amsmath} % for multi - line equations and piecewises
\usepackage{graphicx}
%\usepackage{textcomp}
\usepackage{xspace}
\usepackage{verbatim} % for block comments
%\usepackage{subfig} % for subfigures
\usepackage{enumitem} % for a) b) c) lists
\newcommand{\Cyclus}{\textsc{Cyclus}\xspace} %
\newcommand{\Cycamore}{\textsc{Cycamore}\xspace} %

\usepackage{tabularx}
\usepackage{color}
\usepackage{setspace}
\definecolor{bg}{rgb}{0.95, 0.95, 0.95}
\newcolumntype{b}{X}
\newcolumntype{f}{ > {\hsize=.15\hsize}X}
\newcolumntype{s}{ > {\hsize=.5\hsize}X}
\newcolumntype{m}{ > {\hsize=.75\hsize}X}
\newcolumntype{r}{ > {\hsize=1.1\hsize}X}
\usepackage{titling}
\usepackage[hang, flushmargin]{footmisc}
\renewcommand *\footnoterule{}
\graphicspath{{images /}}
\usepackage[acronym,toc]{glossaries}
\input{../acros}
\usepackage[margin=1in, voffset=0in]{geometry}
\usepackage{authblk}
\makeatletter
\patchcmd{\@maketitle}{\LARGE \@title}{\fontsize{12}{19.2}\selectfont\@title}{}{}
\makeatother


\renewcommand\Authfont{\fontsize{12}{14.4}\selectfont}
\renewcommand\Affilfont{\fontsize{12}{10.8}\itshape}
\setlength{\parindent}{0.25in}



\title{{\textbf{Modeling Material Requirements of the Transition to 
				HALEU Fueled Reactors}}}
\author[1]{Amanda M. Bachmann}
\author[2]{Kathryn Huff }

\affil[1]{\textit{Advanced Reactors and Fuel Cycles, University of Illinois 
at Urbana-Champaign, Department of Nuclear, Plasma, and Radiological 
Engineering, Urbana-Champaign, IL, amandab7@illinois.edu}}
\affil[2]{\textit{Assistant Professor, University of Illinois at 
Urbana-Champaign, Department of Nuclear, Plasma, and Radiological 
Engineering , Urbana-Champaign, IL, 118 Talbot Laboratory, 
kdhuff@illinois.edu
} \vspace{-30pt}}
\date{}


\begin{document}
	\maketitle
	

%Put your content here.
Current nuclear reactors employed in the United States use \gls{LEU} fuel  
enriched to no more than 5\%. New reactor designs, such as the \gls{USNC} 
\gls{MMR}\textsuperscript{TM}, will require \gls{HALEU} fuel enriched 
between 5-20\%. To meet \gls{HALEU} fuel requirements, the U.S. Department 
of Energy is considering recovery and downblending of \gls{HEU} fuel and 
enriching natural uranium to the required levels \cite{griffith_overview_2020}, 
with each of these methods containing their own limitations. The existing 
physical supply of \gls{HEU} and downblending capacity limits the 
recovery and downblending method. Centrifuge capacity limits the 
enrichment of natural uranium method. 
This work aims to quantify the resource requirements 
of the current 
U.S. reactor fleet and of the transition to different reactors that require 
\gls{HALEU} fuel. 

The fuel cycle scenarios are modeled using \Cyclus, an 
agent-based fuel cycle simulator \cite{huff_fundamental_2016}. The scenarios 
model the current U.S. fuel cycle with each reactor unit modeled as a 
separate agent, from 1965 to 2090. The \gls{IAEA} \gls{PRIS} database 
\cite{noauthor_power_1989} provided information about each \gls{LWR} 
in the simulation. 
The first scenario modeled does not include the transition to an advanced 
reactor; is used to provide a baseline of resources required. The next 
scenarios modeled include the transition to either the \gls{USNC} 
\gls{MMR}\textsuperscript{TM}\cite{mitchell_usnc_2020} or the X-energy 
Xe-100\textsuperscript{TM} reactor\cite{hussain_advances_2018}
and model either a no-growth or 1\% growth scenario. This creates five 
different scenarios. The advanced reactors were selected to provide a
comparison between an advanced reactor with small cores and long cycle 
times and an advanced reactor with a large core that employs online 
refueling. 

The results of this work includes the rate of reactor deployment, the mass
of enriched uranium, and the \gls{SWU} capacity required for each 
scenario. These metrics inform the material requirements and provide insight 
into the best method to meet fuel requirements for these transition 
scenarios. Preliminary results show that for the no-growth transition,
5962 \gls{MMR} reactors and 795 Xe-100 reactors are required to meet 
the energy demand, and \gls{MMR} transition requires a greater mass 
of enriched uranium than the Xe-100 transition but requires less 
\gls{SWU} capacity. 

\section*{Acknowledgments}
This material is based upon work supported under an Integrated University 
Program Graduate Fellowship. Any opinions, findings, conclusions, or 
recommendations expressed in this publication are those of the author(s) 
and do not necessarily reflect the views of the Department of Energy Office 
of Nuclear Energy.

Prof. Huff is supported by the Nuclear Regulatory Commission Faculty
Development Program (award NRC-HQ-84-14-G-0054 Program B), the Blue Waters
sustained-petascale computing project supported by the National Science
Foundation (awards OCI-0725070 and ACI-1238993) and the state of Illinois, the
DOE ARPA-E MEITNER Program (award DE-AR0000983), and the DOE H2@Scale Program
(Award Number: DE-EE0008832)


\bibliographystyle{unsrt}
\bibliography{../bibliography}

\end{document}
